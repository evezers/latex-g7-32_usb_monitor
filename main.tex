\documentclass[times]{G7-32} % стиль и язык

\include{preamble.inc} % остальные стандартные настройки убраны в preamble.inc.tex
% Математика
\usepackage{mathtools, cancel, physics, euscript, xfrac, amsmath, amsthm}

\newtheorem{Th}{Теорема}
\newtheorem{Lm}{Лемма}
\newtheorem{Df}{Определение}
\newtheorem{Ex}{Пример}
\newtheorem{Rm}{Ремарка}
\newtheorem{Alg}{Алгоритм}
\newtheorem{Ax}{Аксиома}
\newtheorem{Crl}{Следствие}
\newtheorem{Prp}{Гипотеза}


% Настройки листингов.
\ifPDFTeX
\include{listings.inc}
\else
\usepackage{local-minted}
\fi

\include{macros.inc} % полезные макросы листингов

\include{00-title} % стиль титульного листа и заголовки

\begin{document}

\frontmatter % выключает нумерацию ВСЕГО; здесь начинаются ненумерованные главы: реферат, введение, глоссарий, сокращения и прочее.

\maketitle % создает титульную страницу

\begin{executors}
    \personalSignature{Первый исполнитель}{ФИО}
    \personalSignature{Второй исполнитель}{ФИО}
\end{executors}

%\listoffigures % список рисунков

%\listoftables % список таблиц

%\NormRefs % нормативные ссылки 
% Команды \breakingbeforechapters и \nonbreakingbeforechapters
% управляют разрывом страницы перед главами.
% По-умолчанию страница разрывается.

% \nobreakingbeforechapters
% \breakingbeforechapters

\Referat

Выпускная квалификационная работа магистра состоит из \pageref{LastPage}\,стр.%
\ifnum \totfig >0
, \totfig~рис.%
\fi
\ifnum \tottab >0
, \tottab~табл.%
\fi
%
\ifnum \totbib >0
, \totbib~источн.%
\fi
%
\ifnum \totapp >0
, \totapp~прил.%
\else
.%
\fi

Данный опус, как и более новые версии этого документа, можно взять по адресу (\url{https://github.com/afaikiac/latex-g7-32}). Шаблон соответствует ГОСТ-у настолько хорошо насколько это смогли сделать авторы оригинального репозитория и люди работавшие над этим форком.

Текст в документе носит совершенно абстрактный характер.


\tableofcontents

\printnomenclature % автоматический список сокращений

\Introduction

% Добавлять пункты в "{}Термины и определения"{} и "{}Перечень сокращений и обозначений"{} можно где угодно в коде. Команда \verb|\Define| добавляет пункт в первый список, \verb|\Abbrev| добавляет пункт во второй список.

% Проверяем как у нас работают сокращения, обозначения и определения "---
% MAX,
% \Abbrev{MAX}{maximum "--- максимальное значение параметра}
% API
% \Abbrev{API}{application programming interface "--- внешний интерфейс взаимодействия с приложением}
% с обратным прокси.
% \Define{Обратный прокси}{тип прокси-сервера, который ретранслирует}


\Define{ПЛИС}{программируемая логическая интегральная схема}
\Define{Транзакция}{логическая единица обмена данными}
\Define{Хост}{устройство, являющееся инициатором транзакций на шине}
\Define{UTMI+}{расширение UTMI для поддержки работы в режиме хоста и OTG}
\Abbrev{USB 2.0}{Universal Serial Bus "--- универсальная последовательная шина версии 2.0}
\Abbrev{UTMI}{USB 2.0 Transceiver Macrocell Interface "--- спецификация передатчиков для периферийных устройств, совместимых с интерфейсом USB 2.0}
\Abbrev{OTG}{On-The-Go "--- спецификация мобильных USB-устройств в режиме хоста}
\Abbrev{ULPI}{UTMI+ Low Pin Interface "--- интерфейс UTMI+ с уменьшенным количеством интерфейсных линий}
\Abbrev{ПЛИС}{программируемая логическая интегральная схема "--- позволяет реализовать заказную электронную схему}
\Abbrev{TTL RGB}{от Transistor–transistor logic, также RGBI (Red, Green, Blue, Intensity) "--- цифровой интерфейс, изначально с уровнями транзисторно-транзисторной логики, передающий лишь данные о цветности: красного, зелёного и синего цветов, и яркости освещения}
\Abbrev{LCD}{Liquid Crystal Display "--- технология изготовления жидкокристаллических экранов}
\Abbrev{ПК}{персональный компьютер "--- хост, который будет опрашивать устройство и отправлять ему данные}

Неотъемлемой частью компьютерных систем, взаимодействующих с человеком, являются устройства отображения. Они необходимы для представления информации в графическом виде, более привычном и наглядном для людей. Также важна роль устройств отображения для демонстрации мультимедиа материалов, таких как изображения и видео.

Цель данной научно–исследовательской работы — создание заказного устройства отображения через интерфейс USB на базе ПЛИС.

 Для достижения поставленной цели в работе решаются следующие задачи:

\begin{itemize}
\item ознакомление с интерфейсом USB и ULPI;
\item разработка логики управления контроллером USB и LCD-дисплеем;
\item согласование временных задержек устройств хранения и вывода на экран;
\item тестирование работы устройства.
\end{itemize}

Решению всех перечисленных выше задач и посвящена данная научно–исследовательская работа.


\mainmatter % это включает нумерацию глав и секций в документе ниже

\chapter{Обзор прикладной области}
\label{cha:analysis}

В данном разделе рассматривается прикладная область, необходимая для реализации составных частей устройства.

\section{Организация обмена данными по шине USB}


Шина USB появилась по компьютерным меркам довольно давно "--- версия первого утвержденного варианта стандарта USB 1.0 датируется 15 января 1996 года. Разработка стандарта была инициирована весьма авторитетными фирмами – Intel, DEC, IBM, NEC, Northen Telecom и Compaq \cite{usb:analyse}.

В современных персональных компьютерах USB вытеснил остальные интерфейсы подключения периферийных устройств (такие как COM, LPT, FireWire, \detokenize{PS\2}, 1394) благодаря своей универсальности, открытости и широкой поддержке со стороны производителей оборудования.

Основная цель стандарта, поставленная перед его разработчиками – создать реальную возможность пользователям работать в режиме \detokenize{Plug&Play} с периферийными устройствами. Это означает, что должно быть предусмотрено подключение устройства к работающему компьютеру, автоматическое распознавание его немедленно после подключения и последующей установки соответствующих драйверов. Кроме этого, питание маломощных устройств желательно подавать с самой шины. Скорость шины должна быть достаточной для подавляющего большинства периферийных устройств.

Изохронные передачи USB позволяют передавать цифровые аудиосигналы, а шина USB 2.0 способна нести и видеоданные. Все передачи управляются централизованно, и ПК является необходимым управляющим узлом, находящимся в корне древовидной структуры шины. Адаптер USB входит в состав всех современных чипсетов системных плат \cite{usb:analyse}.

\pagebreak
Технические характеристики:

\begin{itemize}
\item низкая скорость LS (Low Speed USB1.0) – 1,5 Мбит/с;
\item полная скорость FS (Full speed USB1.1) – 12 Мбит/с;
\item высокая скорость HS (High Speed USB2.0) – 480 Мбит/с;
\item максимальное количество подключенных устройств – 127;
\item напряжение питания для периферийных устройств – 5 В;
\item максимальный ток потребления на одно устройство – 100 мА;
\item допустимый ток потребления от ПК по шине USB – 500 мА;
\item шина с использованием промежуточных хабов позволяет соединять устройства, удаленные от компьютера на расстояние до 25 м \cite{usb:specification}.
\end{itemize}


Сигналы USB передаются по 4-х проводному кабелю. Сечения концевых разъемов кабеля приведены на рисунке \ref{fig:usb_connector}. Таблица \ref{tab:usb_signals} содержит пояснения к назначению использования контактов разъемов и проводов кабеля. Здесь GND – цепь «корпуса» для питания периферийных устройств, VBus – +5V также для цепей питания. Шина D+ предназначена для передачи данных по шине, а шина D – для приема данных.

Кабель для поддержки полной скорости шины (full-speed) выполняется как витая пара, защищается экраном и может также использоваться для работы в режиме минимальной скорости (low-speed). Кабель для работы только на минимальной скорости (например, для подключения мыши) может быть любым и неэкранированным \cite{usb:analyse}.

\pagebreak
% \newpage

\begin{figure}[ht!]
    \centering
    \includegraphics[scale=0.5]{res/img/usb_connector.png}
    \caption{Сечения разъемов USB кабеля. A – предназначены только для подключения к источнику, т.е. к компьютеру или хабу; В – предназначены только для подключения к периферийному устройству \cite{usb:specification}}
    \label{fig:usb_connector}
\end{figure}

\begin{table}[ht!]
  \caption{Назначение контактов и проводов USB кабеля \cite{usb:specification}}
    \centering
    \begin{tabular}{|c|c|c|}
    \hline
        \textbf{Номер контакта} & \textbf{Назначение} & \textbf{Цвет провода} \\ \hline
        \textbf{1} & V BUS & Красный \\ \hline
        \textbf{2} & D– & Белый \\ \hline
        \textbf{3} & D+ & Зеленый \\ \hline
        \textbf{4} & GND & Черный \\ \hline
        \textbf{Оплетка} & Экран & Оплетка \\ \hline
  \end{tabular}
  \label{tab:usb_signals}
\end{table}

\pagebreak

\section{Обзор протокола ULPI}

Современный рынок микросхем, поддерживающих обмен данными по шине USB, достаточно обширен, доступны продукты производства фирм Atmel, FTDI, Cypress, Intel, Microchip, National Semiconductor и др.

Для использования в работе была выбрана микросхема Microchip USB3300 по следующим причинам:

\begin{itemize}
\item полная поддержка протокола обмена по стандарту USB2.0;
\item возможность перепрограммирования номеров PID и VID, что позволяет избежать конфликтов при использовании микросхемы в различных устройствах;
\item возможность создания пользовательского устройства, без ограничений по выбору класса;
\item отсутствие накладных расходов на использование протокола, что позволяет полностью использовать пропускную способность USB2.0;
\item популярность и доступность микросхемы;
\item возможность использования стандартизированных UTMI+ блоков с минимальными доработками;
\item наличие документации как на саму микросхему, так и на все уровни протокола.
\end{itemize}

Микросхема USB3300 использует протокол UTMI+ Low Pin Interface (ULPI), позволяющий сократить количество интерфейсных линий, но требующий особого внимания к управлению шиной. В USB3300 протокол ULPI задействован как для настройки конфигурационных параметров приемопередатчика (таких как версия протокола и параметры кодирования), так и для приёма и передачи пакетов USB.

Назначения сигналов интерфейса описаны в таблице \ref{tab:ulpi_signals}.

  \begin{longtable}{|p{0.15\textwidth}|c|p{0.65\textwidth}|}
    \caption{Назначения сигналов интерфейса ULPI \cite{ulpi:usb3300, ulpi:specification}}
    \label{tab:ulpi_signals}
    \\ \hline
    \textbf{Сигнал} & \textbf{Направление} & \textbf{Описание} \\ \hline
    \endfirsthead
    \subcaption{Продолжение таблицы~\ref{tab:ulpi_signals}}
    \\ \hline \endhead
    \hline \subcaption{Продолжение на след. стр.}
    \endfoot
    \hline \endlastfoot
        \textbf{CLKOUT} & выход & \small{Интерфейсный тактовый генератор 60 МГц.} \\ \hline
        \textbf{\detokenize{DATA[7..0]}} & \detokenize{вход\выход} & \small{8-битная двунаправленная шина данных. Принадлежность шины определяется сигналом DIR. Контроллер и устройство инициируют передачу данных путем подачи ненулевого шаблона на шину данных. ULPI определяет синхронизацию интерфейса по нарастающему фронту CLKOUT. DATA\cite{lcd:display} - это старший разряд, а DATA[0] - младший. } \\ \hline
        \textbf{DIR} & выход & \small{Управляет направлением шины данных. DIR подтягивается к высокому уровню контроллером тогда, когда он не должен принимать данные от устройства.} \\ \hline
        \textbf{STP} & вход & \small{Устройство должно подавать сигнал STP, чтобы сигнализировать о завершении передачи пакета USB или операции записи в регистр, а также, по желанию, для остановки приема. Сигнал STP должен быть установлен после последнего байта данных.} \\ \hline
        \textbf{NXT} & выход & \small{Контроллер подает сигнал NXT для задержки отправки данных. Идентичен сигналам RxValid во время приема USB и TxReady во время передачи USB. Контроллер также одновременно подает сигналы NXT и DIR для индикации активности приема USB (RxActive), если до этого шина принадлежала устройству.}
  \end{longtable}

На рисунке \ref{fig:ulpi_dir_change} представлен пример изменения направления шины данных DATA. Каждое изменение владения шиной сопровождается циклом переключения, во время которого управление шиной данных неопределенно. Данные в таком цикле должны быть проигнорированы как контроллером, так и устройством \cite{ulpi:specification}.

\begin{figure}[ht]
    \centering
    \includegraphics[scale=0.5]{res/img/ulpi_dir_change.png}
    \caption{Временная диаграмма изменения направления передачи данных \cite{ulpi:specification}}
    \label{fig:ulpi_dir_change}
\end{figure}

\begin{figure}[ht]
    \centering
    \includegraphics[scale=0.5]{res/img/ulpi_in_out.png}
    \caption{Временная диаграмма приёма и передачи данных \cite{ulpi:specification}}
    \label{fig:ulpi_in_out}
\end{figure}

На рисунке \ref{fig:ulpi_in_out} представлен пример приёма и передачи на шине ULPI: устройство, до этого владевшее шиной данных, устанавливает на шину ненулевые данные, тем самым начиная передачу. По завершении, оно устанавливает STP, указывая на конец передачи. Контроллер перехватывает управление шины, чтобы передать данные устройству. При этом, если больше новых данных нет, владение шиной вновь передаётся устройству \cite{ulpi:specification}.

\section{Интерфейс вывода изображения TTL}
Согласно спецификации дисплейного модуля ccb0702ih40ri-271c, предоставленной в качестве исходных данных для научно-исследовательской работы, его логика основана на интерфейсной микросхеме HX8264, спецификация которой приведена в рекомендуемом источнике \cite{lcd:driver} к заданию на исследовательскую работу. Однако доступ к её конфигурационным линиям не был распаян на шлейфе подключения. Это ограничивает возможности по управлению дисплеем.
Интерфейс передачи данных видеоизображения – TTL схож с протоколом VGA. Временная диаграмма и необходимые параметры задержки приведены на рисунках \ref{fig:ttl_timing} и \ref{fig:lcd_timing} соответственно.

\begin{figure}[ht]
    \centering
    \includegraphics[scale=0.5]{res/img/ttl_timing.png}
    \caption{Временная диаграмма протокола TTL \cite{lcd:display}}
    \label{fig:ttl_timing}
\end{figure}

\begin{figure}[ht]
    \centering
    \includegraphics[scale=0.5]{res/img/lcd_timing.png}
    \caption{Временные характеристики матрицы \cite{lcd:display}}
    \label{fig:lcd_timing}
\end{figure}

\section{Кадровый буфер}

Для синхронизации работы устройства ULPI и видеовыхода необходим буфер, так как тактовые частоты и скорости передачи данных на этих шинах данных не совпадают. Буфер, содержащий кадр перед отправкой его на устройство видеовыхода, называется кадровым буфером.

В качестве кадрового буфера была выбрана память PSRAM, входящая в систему на кристалле ПЛИС Gowin GW1NR-9.

Модуль имеет достаточно простой интерфейс: шина адреса, шина данных и управляющие сигналы разрешения передачи данных и \detokenize{«data_mask»}, с помощью которго можно защитить часть данных от записи. В частности, сигнал «cmd» "--- это тип команды: 1 - запись, 0 - чтение. Сигнал \detokenize{«cmd_en»} обозначает момент начала ее выполнения. Все операции пакетные и длина пакета зависит от режима Burst Mode. Начинается пакет записываемых данных в один момент с началом команды. Операция записи в режиме Burst Mode - 16 показана на рисунке \ref{fig:psram_write}.

\begin{figure}[ht]
    \centering
    \includegraphics[scale=0.5]{res/img/psram_write.png}
    \caption{Временная диаграмма операции записи \cite{gowin:GW1NSR}}
    \label{fig:psram_write}
\end{figure}

\begin{figure}[ht]
    \centering
    \includegraphics[scale=0.5]{res/img/psram_read.png}
    \caption{Временная диаграмма операции чтения \cite{gowin:GW1NSR}}
    \label{fig:psram_read}
\end{figure}

Операция чтения обозначена на рисунке \ref{fig:psram_read}. Читаемые данные появляются на шине не сразу, а через некоторое количество тактов. Это количество всегда постоянно и равно 16 тактам. Весь пакет данных из памяти сопровождается сигналом «\detokenize{rd_data_valid}».

\section{Выводы}

\begin{packed_enum}
\item Изучены рекомендуемые источники и выявлены другие источники.
\item Намечены разделы обзора и составлен (написан) сам обзор по теме НИР.
\item Обоснован выбор компонентной базы устройства отображения.
\item Описана работа кадрового буфера.
\item Представлены основные протоколы и интерфейсы устройств, используемых в данной работе.
\end{packed_enum}

\nocite{vhdl:specification}

\chapter{Конструкторский раздел}
\label{cha:design}

В данном разделе проектируется новая всячина.

\section{Архитектура всячины}

Используйте окружение \verb|figure| для создания изображений. Код ниже является стандартным и его следует копировать каждый раз, когда необходимо вставить новое изображение.

Команда \verb|\includegraphics|, находящаяся внутри окружения, непосредственно вставляет изображение, единственный ее аргумент "--- путь до изображения. Используйте параметр \verb|scale| для изменения размера изображения. Команда \verb|\caption| задает подпись под изображением, а команда \verb|\label| задает уникальный идентификатор объекта, на который можно ссылаться в тексте.

\begin{figure}[ht]
    \centering
    \includegraphics[scale=0.5]{example-image-a}
    \caption{Подпись к изображению}
    \label{fig:example_fig_1}
\end{figure}

\begin{figure}[ht]
    \centering
    \includegraphics[scale=1.5,angle=90]{example-image}
    \caption{Большой рисунок}
\end{figure}

Кстати, про изображения. Во-первых, для изображений следует использовать \verb|[ht]|. Если после этого изображения все еще вставляются "{}не по ГОСТ"{}, т.е. слишком далеко от места ссылки, "--- значит, у вас \textbf{слишком мало текста}! Хотя и ужасный параметр \verb|ht!| у окружения \texttt{figure} тоже никто не отменял, но его использование делает документ страшным, как в Word. Поэтому просьба не использовать его, если это возможно.

\section{Существующие подходы к созданию всячины}

В отчетах может быть необходимо использовать таблицы "--- см. табл.~\ref{tab:tabular} и~\ref{tab:longtable}. Для создания небольших таблиц можно использовать окружение \verb|tabular| внутри окружения \verb|table| (последнее полностью аналогично окружению \verb|figure|, но добавляет другую подпись). Для более продвинутых таблиц можно обратиться к информации на сайте \url{https://www.overleaf.com/learn/latex/Tables}. Там вы найдете советы по созданию сложных таблиц, но они не слишком сложные.

\begin{table}[ht]
  \caption{Пример короткой таблицы с коротким названием}
  \begin{tabular}{|r|c|c|c|l|}
  \hline
  Тело      & $F$ & $V$  & $E$ & $F+V-E-2$ \\
  \hline
  Тетраэдр  & 4   & 4    & 6   & 0         \\
  Куб       & 6   & 8    & 12  & 0         \\
  Октаэдр   & 8   & 6    & 12  & 0         \\
  Додекаэдр & 20  & 12   & 30  & 0         \\
  Икосаэдр  & 12  & 20   & 30  & 0         \\
  \hline
  Эйлер     & 666 & 9000 & 42  & $+\infty$ \\
  \hline
  \end{tabular}
  \label{tab:tabular}
\end{table}

Для создания больших таблиц следует использовать пакет \verb|longtable|, который позволяет создавать таблицы на несколько страниц в соответствии с требованиями ГОСТ.

Для того, чтобы длинный текст разбивался на несколько строк в пределах одной ячейки, необходимо использовать формат ячейки \texttt{p} и явно указывать ее ширину в мм/дюймах (\texttt{110mm}), относительно ширины страницы (\texttt{0.22\textbackslash textwidth}), и т.п.

Можно также использовать уменьшенный шрифт, но, пожалуйста, не забывайте применять его ко всей таблице сразу.

\begin{center}
  \begin{longtable}{|p{0.40\textwidth}|c|p{0.30\textwidth}|}
    \caption{Пример длинной таблицы с длинным названием на много длинных-длинных строк}
    \label{tab:longtable}
    \\ \hline
    Вид шума & Громкость, дБ & Комментарий \\
    \hline \endfirsthead
    \subcaption{Продолжение таблицы~\ref{tab:longtable}}
    \\ \hline \endhead
    \hline \subcaption{Продолжение на след. стр.}
    \endfoot
    \hline \endlastfoot
    Порог слышимости             & 0     &                                                \\
    \hline
    Шепот в тихой библиотеке     & 30    &                                                \\
    Обычный разговор             & 60-70 &                                                \\
    Звонок телефона              & 80    & \small{Конечно, это было до эпохи мобильников} \\
    Уличный шум                  & 85    & \small{(внутри машины)}                        \\
    Гудок поезда                 & 90    &                                                \\
    Шум электрички               & 95    &                                                \\
    \hline
    Порог здоровой нормы         & 90-95 & \small{Длительное пребывание на более
    громком шуме может привести к ухудшению слуха}                                        \\
    \hline
    Мотоцикл                     & 100   &                                                \\
    Power Mower                  & 107   & \small{(модель бензокосилки)}                  \\
    Бензопила                    & 110   & \small{(Doom в целом вреден для здоровья)}     \\
    Рок-концерт                  & 115   &                                                \\
    \hline
    Порог боли                   & 125   & \small{feel the pain}                          \\
    \hline
    Клепальный молоток           & 125   & \small{(автор сам не знает, что это)}          \\
    \hline
    Порог опасности              & 140   & \small{Даже кратковременное пребывание на
    шуме большего уровня может привести к необратимым последствиям}                       \\
    \hline
    Реактивный двигатель         & 140   &                                                \\
                                 & 180   & \small{Необратимое полное повреждение
                                 слуховых органов}                                        \\
    Самый громкий возможный звук & 194   & \small{Интересно, почему?..}                   \\
  \end{longtable}
\end{center}

\chapter{Технологический раздел}
\label{cha:impl}

В данном разделе описано изготовление и требование всячины. Кстати,
в Latex нужно эскейпить подчёркивание (писать <<\verb|some\_function|>> для \Code{some\_function}).

\ifPDFTeX
% TODO: сделать рабочее решение для листингов c русским текстом в pdflatex
\else

Для вставки кода есть пакет \texttt{minted}. Он хорош всем кроме: необходимости Python (есть во всех нормальных (нет, Windows, я не про тебя) ОС) и Pygments и того, что нормально работает лишь в \XeLaTeX.

\ifdefined\NoMinted
Но к сожалению, у вас, по-видимому, не установлен Python или pygmentize.
\else
Можно пользоваться расширенным BFN:

\begin{listing}[H]
\begin{ebnfcode}
 letter = "A" | "B" | "C" | "D" | "E" | "F" | "G"
       | "H" | "I" | "J" | "K" | "L" | "M" | "N"
       | "O" | "P" | "Q" | "R" | "S" | "T" | "U"
       | "V" | "W" | "X" | "Y" | "Z" ;
digit = "0" | "1" | "2" | "3" | "4" | "5" | "6" | "7" | "8" | "9" ;
symbol = "[" | "]" | "{" | "}" | "(" | ")" | "<" | ">"
       | "'" | '"' | "=" | "|" | "." | "," | ";" ;
character = letter | digit | symbol | "_" ;
 
identifier = letter , { letter | digit | "_" } ;
terminal = "'" , character , { character } , "'" 
         | '"' , character , { character } , '"' ;
 
lhs = identifier ;
rhs = identifier
     | terminal
     | "[" , rhs , "]"
     | "{" , rhs , "}"
     | "(" , rhs , ")"
     | rhs , "|" , rhs
     | rhs , "," , rhs ;
 
rule = lhs , "=" , rhs , ";" ;
grammar = { rule } ;
\end{ebnfcode}
\caption{EBNF определённый через EBNF}
\label{lst:ebnf}
\end{listing}

А вот в листинге \ref{lst:c} на языке C работают русские комменты. Спасибо Pygments и Minted за это.

\begin{listing}[H]
\cfile{res/code/test.c}
\caption{Пример — test.c} 
\end{listing}
\label{lst:c}

\fi
\fi

\chapter{Экспериментальный раздел}
\label{cha:research}

Математическая формула может встречаться в тексте: $E = mc^2$. Для больших формул следует использовать окружение \verb|equation|:
\begin{equation}\label{eq:f1}
    E = mc^2.
\end{equation}
На формулу можно сослаться, например, так: (\ref{eq:f1}). См. также раздел \ref{cha:econom}. Использование других окружений для формул, таких как двойные доллары или скобки:
\[
    E = mc^2,
\]
не рекомендуется из-за отсутствия нумерации.

В конце больших формул следует ставить подходящий знак препинания в соответствии с контекстом: точку, запятую, точку с запятой или ничего.

Согласно ГОСТ, каждое новое обозначение, вводимое в формуле, должно быть пояснено сразу после неё. Например:
\begin{equation}
    E = mc^2,
\end{equation}
где $m$ "--- масса, $c$ "--- скорость света.

Несколько примеров:

Формула с текстом:
\begin{equation}
    50 \text{ яблок} \times 100 \text{ яблок} =
    \textbf{ много яблок}
\end{equation}

Различные буквы и шрифты:
\begin{equation}
    \alpha,  \beta,  \gamma, \Gamma, \pi, \Pi, \phi, \varphi, \mu, \Phi, \xi, \zeta;
\end{equation}
\begin{equation}
    \mathbf M, \mathcal C, \mathbb R, \sin \theta = \mathrm{sin} \theta.
\end{equation}

Скобки:
\begin{equation}
    ( a ), [ b ], \{ c \}, | d |, \| e \|, \langle f \rangle, \lfloor g \rfloor, \lceil h \rceil, \ulcorner i \urcorner;
\end{equation}
\begin{equation}
    \left( a + b \right) \left[ 1 - \frac{b}{a+b} \right] = a;
\end{equation}
\begin{equation}
    \sqrt{|xy|} \leq \left| \frac{x + y}{2} \right|;
\end{equation}
\begin{equation}
    \int_a^b u \dv[2]{v}{x} \dd x = \left. u \dv{v}{x} \right|_a^b -\int_a^b \dv{u}{x} \dv{v}{x} \dd x;
\end{equation}
\begin{equation}
    \tilde f(\omega) = \frac{1}{2\pi} \int_{-\infty}^\infty f(x)e^{-i\omega x} \dd x;
\end{equation}
\begin{equation}
    \dot{\vec \omega} = \vec r_c \times \vec I;
\end{equation}
\begin{equation}
    u = \frac{-y}{x^2 + y^2}, \quad v = \frac{x}{x^2 + y^2}, \quad \text{и} \quad w = 0.
\end{equation}

Последовательности:
\begin{equation}
    (1+x)^n = \sum_{i=0}^n \binom{n}{i} x^i;
\end{equation}
\begin{equation}
    e^x = 1 + x + \frac{x^2}{2} + \frac{x^3}{6} + \cdots = \sum_{n \ge 0} \frac{x^n}{n!};
\end{equation}

Дроби:
\begin{equation}
    x = 
    a_0 + \frac{1}{a_1 + \frac{1}{a_2 + \frac{1}{a_3 + a_4}}}
    =
    a_0 + \frac{1}{\displaystyle a_1
        + \frac{1}{\displaystyle a_2
        + \frac{1}{\displaystyle a_3 + a_4}}}.
\end{equation}

Матрицы:
\begin{equation}
    \begin{pmatrix}
        1 & x & 0 \\
        0 & 1 & -1
    \end{pmatrix}
    \begin{pmatrix}
        1  \\
        y  \\
        1
    \end{pmatrix}
    =
    \begin{pmatrix}
        1 + xy  \\
        y - 1
    \end{pmatrix},
    \quad\quad
    \left(
    \begin{matrix}
        2 & 3 & 4\\
        5 & 6 & 7\\
        8 & 9 & 10
    \end{matrix}
    \right)
    v = 0;
\end{equation}
\begin{equation}
    \frac{n!}{k!(n-k)!} = \binom{n}{k};
\end{equation}
\begin{equation}
    \deg A =
    \left|
    \begin{matrix}
        -2 & 1 & 0 & 0 & \cdots & 0  \\
        1 & -2 & 1 & 0 & \cdots & 0  \\
        0 & 1 & -2 & 1 & \cdots & 0  \\
        0 & 0 & 1 & -2 & \ddots & \vdots \\
        \vdots & \vdots & \vdots & \ddots & \ddots & 1  \\
        0 & 0 & 0 & \cdots & 1 & -2
    \end{matrix}
    \right|.
\end{equation}

Фигурная скобка для нескольких случаев:
\begin{equation}
    |x| =
    \begin{cases}
        x, & x \ge 0, \\
        -x, & x< 0.
    \end{cases}
\end{equation}

Формулы в несколько строк:
\begin{align}
    F &= \{ F_{x} \in F_{c} \mid (|S| > |C|) \\
      &\wedge (\mathrm{minPixels} < |S| < \mathrm{maxPixels}) \\
      &\wedge (|S_{\mathrm{conected}}| > |S| - \epsilon) \}
\end{align}
или
\begin{multline}
    A_0 = \frac{1}{(\alpha + t_x)^{r + s + x}}{}_2 F_1 \left( r + s + x, x + 1; r + s + x + 1; \frac{\alpha - \beta}{\alpha + t_x} \right) \\
    \quad - \frac{1}{(\alpha + T)^{r + s + x}}{}_2 F_1 \left( r + s + x, x + 1; r + s + x + 1; \frac{\alpha - \beta}{\alpha + T} \right).
\end{multline}

Логика, доказательства:
\begin{equation}
    (\forall \varepsilon > 0) (\exists N \in \mathbb Z^+) (\forall n \ge N) (|x_n - a| < \varepsilon \iff \lim_{n \to +\infty} x_n = a);
\end{equation}
\begin{equation}
    A \implies B, \quad A \iff B, \quad A = \{z \in \mathbb Z \mid z = \bar z \};
\end{equation}
\begin{equation}
    f: X \to Y, \quad f: x \overset{F}{\mapsto} 5x \cos(\tfrac{\pi x}{2})
\end{equation}
\begin{equation}
    \frac{\cancel{\sqrt 2} \sin(x + 2)}{\cancel{2 \cdot \cos(\sfrac{\pi}{4})} \sin(x)} = \frac{\sin(x + 2)}{\sin(x)}, \quad \cancelto{0}{\sin(0)} \equiv 0.
\end{equation}

\section{Определения и теоремы}

В ГОСТе не представлены инструкции по оформлению определений и теорем. Однако в мировой практике принято следующее стилевое решение, которое можно использовать и в дипломной работе:

\begin{Df}[Метрическое пространство]\label{df:example}
    Метрическим пространством называют пару $(S; \rho)$, где для функции-\textit{метрики} $\rho: S \to \mathbb R^+$, если верно следующие три условия:
    \begin{enumerate}
        \item $a \in S : \rho(a, a) = 0$,
        \item $a, b \in S : \rho(a, b) = \rho(b, a)$,
        \item $a, b, c \in S : \rho(a, b) + \rho(b, c) \ge \rho(a, c)$.
    \end{enumerate}
\end{Df}

\begin{Th}[Хаусдорф]\label{th:example}
    В полном метрическом пространстве множество является компактом тогда и только тогда, когда оно замкнуто и вполне ограничено.
\end{Th}
\begin{proof}
    Текст доказательства.
\end{proof}

\begin{Ex}
    Множество вещественных чисел $\mathbb R$ с заданной на нём метрикой $\rho(a, b) = |a - b|$ формируют метрическое пространство.
\end{Ex}
\begin{proof}
    Следует из определения \ref{df:example}.
\end{proof}

Аналогично картинкам и таблицам, на определения, теоремы и другие блоки можно ссылаться при помощи команды \verb|\ref|: например, на теорему \ref{th:example}.

\chapter{Организационно-экономический раздел}
\label{cha:econom}

Интернет-ссылки оформляются как \url{https://www.google.com/}. Ссылка на картинку, формулу и другие объекты происходит при помощи команды \verb|\ref|: см. рис. \ref{fig:example_fig_1}. На ссылку можно кликнуть и перейти к объекту, на который она ссылается. Ссылка может быть как до объекта, так и после него. Обычно, для удобства, идентификатор начинают с префикса <<\verb|fig:|>> для рисунков, <<\verb|tab:|>> для таблиц, <<\verb|eq:|>> для формул, для <<\verb|section:|>> для разделов, <<\verb|th:|>> для теорем и так далее. Пример ссылки на раздел: см. раздел \ref{cha:research}. 

По условиям ГОСТ-а, на каждую картинку и таблицу должна присутствовать в тексте хотя бы одна ссылка. Желательно рядом с картинкой. Ссылка должна быть оформлена в духе "{}В соответствии с рисунком (таблицей, разделом) 2"{}.

Информация об цитируемых источниках хранится в файлах формата \verb|*.bib|. В этом проекте есть пример такого файла \verb|references.bib|. Похож на формат \verb|json|, но им не является. Заполняйте как можно больше полей там. Сразу готовую запись можно получить прямо в Google Scholar или, часто, на страницах статей на других ресурсах (см. рис. \ref{fig:scholar}). Описание статей начинается с заголовка \verb|@article|, для источников других видов предусмотрены другие заголовки, например \verb|@online| для интернет-страниц.

\begin{figure}[ht]
    \centering
    \includegraphics[scale=0.5]{res/img/scholar.png}
    \caption{Как получить bib-цитату на источник в Google Scholar.}
    \label{fig:scholar}
\end{figure}

Ссылка на источник происходит при помощи команды \verb|\cite|: \cite{duportail:alu}. В качестве единственного аргумента указывается идентификатор источника в одном из файлов \verb|*.bib|. Можно указать несколько источников: \cite{duportail:alu, husserl:pd, althusser:iia}, с \verb|\ref| так нельзя. В списке источников отображаются только те источники, на которых есть хотя бы одна ссылка. Если всё-же нужно что бы он там появился без единой ссылки, можно использовать команду \verb|\nocite| в любом месте программы, как под этим абзацем.

Вообще все ссылки кликабельны. Если ссылка неправильная, компилятор выдаст предупреждение, а ссылка будет выглядеть так: \textbf{??}.

\nocite{duportail:alu}
\nocite{althusser:iia}
\nocite{husserl:pd}
\nocite{husserl:sbe}

\include{15-bzd}

\backmatter % здесь заканчивается нумерованная часть документа и начинаются ссылки и
            
\Conclusion

В результате проделанной работы стало ясно, что ничего не ясно...
 % заключение

\bibliographystyle{ugost2008}
\bibliography{res/references.bib}


\appendix % тут идут приложения

\include{30-appendix1}

\chapter{Еще картинки}
\label{cha:appendix2}

\blindtext

\begin{figure}
    \centering
    \includegraphics{example-image-golden}
    \caption{Еще одна картинка, ничем не лучше предыдущей. Но надо же как-то заполнить место}
\end{figure}


\end{document}
