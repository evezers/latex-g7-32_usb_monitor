\chapter{Конструкторский раздел}
\label{cha:design}

В данном разделе проектируется новая всячина.

\section{Архитектура всячины}

Используйте окружение \verb|figure| для создания изображений. Код ниже является стандартным и его следует копировать каждый раз, когда необходимо вставить новое изображение.

Команда \verb|\includegraphics|, находящаяся внутри окружения, непосредственно вставляет изображение, единственный ее аргумент "--- путь до изображения. Используйте параметр \verb|scale| для изменения размера изображения. Команда \verb|\caption| задает подпись под изображением, а команда \verb|\label| задает уникальный идентификатор объекта, на который можно ссылаться в тексте.

\begin{figure}[ht]
    \centering
    \includegraphics[scale=0.5]{example-image-a}
    \caption{Подпись к изображению}
    \label{fig:example_fig_1}
\end{figure}

\begin{figure}[ht]
    \centering
    \includegraphics[scale=1.5,angle=90]{example-image}
    \caption{Большой рисунок}
\end{figure}

Кстати, про изображения. Во-первых, для изображений следует использовать \verb|[ht]|. Если после этого изображения все еще вставляются "{}не по ГОСТ"{}, т.е. слишком далеко от места ссылки, "--- значит, у вас \textbf{слишком мало текста}! Хотя и ужасный параметр \verb|ht!| у окружения \texttt{figure} тоже никто не отменял, но его использование делает документ страшным, как в Word. Поэтому просьба не использовать его, если это возможно.

\section{Существующие подходы к созданию всячины}

В отчетах может быть необходимо использовать таблицы "--- см. табл.~\ref{tab:tabular} и~\ref{tab:longtable}. Для создания небольших таблиц можно использовать окружение \verb|tabular| внутри окружения \verb|table| (последнее полностью аналогично окружению \verb|figure|, но добавляет другую подпись). Для более продвинутых таблиц можно обратиться к информации на сайте \url{https://www.overleaf.com/learn/latex/Tables}. Там вы найдете советы по созданию сложных таблиц, но они не слишком сложные.

\begin{table}[ht]
  \caption{Пример короткой таблицы с коротким названием}
  \begin{tabular}{|r|c|c|c|l|}
  \hline
  Тело      & $F$ & $V$  & $E$ & $F+V-E-2$ \\
  \hline
  Тетраэдр  & 4   & 4    & 6   & 0         \\
  Куб       & 6   & 8    & 12  & 0         \\
  Октаэдр   & 8   & 6    & 12  & 0         \\
  Додекаэдр & 20  & 12   & 30  & 0         \\
  Икосаэдр  & 12  & 20   & 30  & 0         \\
  \hline
  Эйлер     & 666 & 9000 & 42  & $+\infty$ \\
  \hline
  \end{tabular}
  \label{tab:tabular}
\end{table}

Для создания больших таблиц следует использовать пакет \verb|longtable|, который позволяет создавать таблицы на несколько страниц в соответствии с требованиями ГОСТ.

Для того, чтобы длинный текст разбивался на несколько строк в пределах одной ячейки, необходимо использовать формат ячейки \texttt{p} и явно указывать ее ширину в мм/дюймах (\texttt{110mm}), относительно ширины страницы (\texttt{0.22\textbackslash textwidth}), и т.п.

Можно также использовать уменьшенный шрифт, но, пожалуйста, не забывайте применять его ко всей таблице сразу.

\begin{center}
  \begin{longtable}{|p{0.40\textwidth}|c|p{0.30\textwidth}|}
    \caption{Пример длинной таблицы с длинным названием на много длинных-длинных строк}
    \label{tab:longtable}
    \\ \hline
    Вид шума & Громкость, дБ & Комментарий \\
    \hline \endfirsthead
    \subcaption{Продолжение таблицы~\ref{tab:longtable}}
    \\ \hline \endhead
    \hline \subcaption{Продолжение на след. стр.}
    \endfoot
    \hline \endlastfoot
    Порог слышимости             & 0     &                                                \\
    \hline
    Шепот в тихой библиотеке     & 30    &                                                \\
    Обычный разговор             & 60-70 &                                                \\
    Звонок телефона              & 80    & \small{Конечно, это было до эпохи мобильников} \\
    Уличный шум                  & 85    & \small{(внутри машины)}                        \\
    Гудок поезда                 & 90    &                                                \\
    Шум электрички               & 95    &                                                \\
    \hline
    Порог здоровой нормы         & 90-95 & \small{Длительное пребывание на более
    громком шуме может привести к ухудшению слуха}                                        \\
    \hline
    Мотоцикл                     & 100   &                                                \\
    Power Mower                  & 107   & \small{(модель бензокосилки)}                  \\
    Бензопила                    & 110   & \small{(Doom в целом вреден для здоровья)}     \\
    Рок-концерт                  & 115   &                                                \\
    \hline
    Порог боли                   & 125   & \small{feel the pain}                          \\
    \hline
    Клепальный молоток           & 125   & \small{(автор сам не знает, что это)}          \\
    \hline
    Порог опасности              & 140   & \small{Даже кратковременное пребывание на
    шуме большего уровня может привести к необратимым последствиям}                       \\
    \hline
    Реактивный двигатель         & 140   &                                                \\
                                 & 180   & \small{Необратимое полное повреждение
                                 слуховых органов}                                        \\
    Самый громкий возможный звук & 194   & \small{Интересно, почему?..}                   \\
  \end{longtable}
\end{center}
