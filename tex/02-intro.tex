\Introduction

% Добавлять пункты в "{}Термины и определения"{} и "{}Перечень сокращений и обозначений"{} можно где угодно в коде. Команда \verb|\Define| добавляет пункт в первый список, \verb|\Abbrev| добавляет пункт во второй список.

% Проверяем как у нас работают сокращения, обозначения и определения "---
% MAX,
% \Abbrev{MAX}{maximum "--- максимальное значение параметра}
% API
% \Abbrev{API}{application programming interface "--- внешний интерфейс взаимодействия с приложением}
% с обратным прокси.
% \Define{Обратный прокси}{тип прокси-сервера, который ретранслирует}


\Define{ПЛИС}{программируемая логическая интегральная схема}
\Define{Транзакция}{логическая единица обмена данными}
\Define{Хост}{устройство, являющееся инициатором транзакций на шине}
\Define{UTMI+}{расширение UTMI для поддержки работы в режиме хоста и OTG}
\Abbrev{USB 2.0}{Universal Serial Bus "--- универсальная последовательная шина версии 2.0}
\Abbrev{UTMI}{USB 2.0 Transceiver Macrocell Interface "--- спецификация передатчиков для периферийных устройств, совместимых с интерфейсом USB 2.0}
\Abbrev{OTG}{On-The-Go "--- спецификация мобильных USB-устройств в режиме хоста}
\Abbrev{ULPI}{UTMI+ Low Pin Interface "--- интерфейс UTMI+ с уменьшенным количеством интерфейсных линий}
\Abbrev{ПЛИС}{программируемая логическая интегральная схема "--- позволяет реализовать заказную электронную схему}
\Abbrev{TTL RGB}{от Transistor–transistor logic, также RGBI (Red, Green, Blue, Intensity) "--- цифровой интерфейс, изначально с уровнями транзисторно-транзисторной логики, передающий лишь данные о цветности: красного, зелёного и синего цветов, и яркости освещения}
\Abbrev{LCD}{Liquid Crystal Display "--- технология изготовления жидкокристаллических экранов}
\Abbrev{ПК}{персональный компьютер "--- хост, который будет опрашивать устройство и отправлять ему данные}

Неотъемлемой частью компьютерных систем, взаимодействующих с человеком, являются устройства отображения. Они необходимы для представления информации в графическом виде, более привычном и наглядном для людей. Также важна роль устройств отображения для демонстрации мультимедиа материалов, таких как изображения и видео.

Цель данной научно–исследовательской работы — создание заказного устройства отображения через интерфейс USB на базе ПЛИС.

 Для достижения поставленной цели в работе решаются следующие задачи:

\begin{itemize}
\item ознакомление с интерфейсом USB и ULPI;
\item разработка логики управления контроллером USB и LCD-дисплеем;
\item согласование временных задержек устройств хранения и вывода на экран;
\item тестирование работы устройства.
\end{itemize}

Решению всех перечисленных выше задач и посвящена данная научно–исследовательская работа.
