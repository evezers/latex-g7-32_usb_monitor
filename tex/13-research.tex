\chapter{Экспериментальный раздел}
\label{cha:research}

\section{Практическая}

Проверка:


\begin{algorithm}\label{algo:ProclivityPro
p2}
\caption{Graph}
\begin{algorithmic}
  \Input
  \Desc{T}{matrix of measurements}
  \EndInput
  \Output
  \Desc{$\hat X$}{matrix of graph signals}
  \Desc{$\hat W$}{matrix of outliers}
  \EndOutput
\end{algorithmic}
\end{algorithm}


\begin{tikzpicture}
    \matrix (m) [
        matrix of nodes,
        nodes in empty cells,
        nodes = {
            text height=2ex, text depth=0.5ex,
            inner sep=1mm, outer sep=0mm,
        },
        row sep = 0mm, column sep = 7mm,
        column 1/.style = {nodes={align=right, anchor=south east}},
        column 2/.style = {nodes={align=center, anchor=south, text width=18ex}}, % change width of block here
        column 3/.style = {nodes={align=left, anchor=south west}},
    ] {
    clk             &                   &   ctrl\_tready    \\
    rst             &                   &   output\_tvalid  \\
    input\_tvalid   &                   &   output\_tlast   \\
    input\_tdata    &   conv\_encoderjk &                   \\
    input\_tlast    &                   &   output\_tdata   \\
                    &                   &   test1           \\
    ctrl\_tvalid    &                   &                   \\
    ctrl\_tvalid    &                   &                   \\
    ctrl\_tvalid    &                   &                   \\
    ctrl\_tvalid    &                   &                   \\
    ctrl\_tvalid    &                   &   jjjjj           \\
    };
    \scoped[on background layer]
        \node (enc)  [draw, rounded corners, semithick, fill=gray!10,
                      inner sep = 0mm, outer sep= 0mm,
                      fit=(m-1-2) (m-11-2)] {};
    \foreach    \i in {1,...,5,7}
        \draw[-{Triangle[angle=60:2pt 4]}]    (m-\i-1) -- (m-\i-2);
    \foreach    \i in {1,...,3,5}
        \draw[-{Triangle[angle=60:2pt 4]}]    (m-\i-2) -- (m-\i-3);
    \foreach    \i in {6}
        \draw[{Stealth[scale=2]}-]    (m-\i-2) -- (m-\i-3);
\end{tikzpicture}


Математическая формула может встречаться в тексте: $E = mc^2$. Для больших формул следует использовать окружение \verb|equation|:
\begin{equation}\label{eq:f1}
    E = mc^2.
\end{equation}
На формулу можно сослаться, например, так: (\ref{eq:f1}). См. также раздел \ref{cha:econom}. Использование других окружений для формул, таких как двойные доллары или скобки:
\[
    E = mc^2,
\]
не рекомендуется из-за отсутствия нумерации.

В конце больших формул следует ставить подходящий знак препинания в соответствии с контекстом: точку, запятую, точку с запятой или ничего.

Согласно ГОСТ, каждое новое обозначение, вводимое в формуле, должно быть пояснено сразу после неё. Например:
\begin{equation}
    E = mc^2,
\end{equation}
где $m$ "--- масса, $c$ "--- скорость света.

Несколько примеров:

Формула с текстом:
\begin{equation}
    50 \text{ яблок} \times 100 \text{ яблок} =
    \textbf{ много яблок}
\end{equation}

Различные буквы и шрифты:
\begin{equation}
    \alpha,  \beta,  \gamma, \Gamma, \pi, \Pi, \phi, \varphi, \mu, \Phi, \xi, \zeta;
\end{equation}
\begin{equation}
    \mathbf M, \mathcal C, \mathbb R, \sin \theta = \mathrm{sin} \theta.
\end{equation}

Скобки:
\begin{equation}
    ( a ), [ b ], \{ c \}, | d |, \| e \|, \langle f \rangle, \lfloor g \rfloor, \lceil h \rceil, \ulcorner i \urcorner;
\end{equation}
\begin{equation}
    \left( a + b \right) \left[ 1 - \frac{b}{a+b} \right] = a;
\end{equation}
\begin{equation}
    \sqrt{|xy|} \leq \left| \frac{x + y}{2} \right|;
\end{equation}
\begin{equation}
    \int_a^b u \dv[2]{v}{x} \dd x = \left. u \dv{v}{x} \right|_a^b -\int_a^b \dv{u}{x} \dv{v}{x} \dd x;
\end{equation}
\begin{equation}
    \tilde f(\omega) = \frac{1}{2\pi} \int_{-\infty}^\infty f(x)e^{-i\omega x} \dd x;
\end{equation}
\begin{equation}
    \dot{\vec \omega} = \vec r_c \times \vec I;
\end{equation}
\begin{equation}
    u = \frac{-y}{x^2 + y^2}, \quad v = \frac{x}{x^2 + y^2}, \quad \text{и} \quad w = 0.
\end{equation}

Последовательности:
\begin{equation}
    (1+x)^n = \sum_{i=0}^n \binom{n}{i} x^i;
\end{equation}
\begin{equation}
    e^x = 1 + x + \frac{x^2}{2} + \frac{x^3}{6} + \cdots = \sum_{n \ge 0} \frac{x^n}{n!};
\end{equation}

Дроби:
\begin{equation}
    x = 
    a_0 + \frac{1}{a_1 + \frac{1}{a_2 + \frac{1}{a_3 + a_4}}}
    =
    a_0 + \frac{1}{\displaystyle a_1
        + \frac{1}{\displaystyle a_2
        + \frac{1}{\displaystyle a_3 + a_4}}}.
\end{equation}

Матрицы:
\begin{equation}
    \begin{pmatrix}
        1 & x & 0 \\
        0 & 1 & -1
    \end{pmatrix}
    \begin{pmatrix}
        1  \\
        y  \\
        1
    \end{pmatrix}
    =
    \begin{pmatrix}
        1 + xy  \\
        y - 1
    \end{pmatrix},
    \quad\quad
    \left(
    \begin{matrix}
        2 & 3 & 4\\
        5 & 6 & 7\\
        8 & 9 & 10
    \end{matrix}
    \right)
    v = 0;
\end{equation}
\begin{equation}
    \frac{n!}{k!(n-k)!} = \binom{n}{k};
\end{equation}
\begin{equation}
    \deg A =
    \left|
    \begin{matrix}
        -2 & 1 & 0 & 0 & \cdots & 0  \\
        1 & -2 & 1 & 0 & \cdots & 0  \\
        0 & 1 & -2 & 1 & \cdots & 0  \\
        0 & 0 & 1 & -2 & \ddots & \vdots \\
        \vdots & \vdots & \vdots & \ddots & \ddots & 1  \\
        0 & 0 & 0 & \cdots & 1 & -2
    \end{matrix}
    \right|.
\end{equation}

Фигурная скобка для нескольких случаев:
\begin{equation}
    |x| =
    \begin{cases}
        x, & x \ge 0, \\
        -x, & x< 0.
    \end{cases}
\end{equation}

Формулы в несколько строк:
\begin{align}
    F &= \{ F_{x} \in F_{c} \mid (|S| > |C|) \\
      &\wedge (\mathrm{minPixels} < |S| < \mathrm{maxPixels}) \\
      &\wedge (|S_{\mathrm{conected}}| > |S| - \epsilon) \}
\end{align}
или
\begin{multline}
    A_0 = \frac{1}{(\alpha + t_x)^{r + s + x}}{}_2 F_1 \left( r + s + x, x + 1; r + s + x + 1; \frac{\alpha - \beta}{\alpha + t_x} \right) \\
    \quad - \frac{1}{(\alpha + T)^{r + s + x}}{}_2 F_1 \left( r + s + x, x + 1; r + s + x + 1; \frac{\alpha - \beta}{\alpha + T} \right).
\end{multline}

Логика, доказательства:
\begin{equation}
    (\forall \varepsilon > 0) (\exists N \in \mathbb Z^+) (\forall n \ge N) (|x_n - a| < \varepsilon \iff \lim_{n \to +\infty} x_n = a);
\end{equation}
\begin{equation}
    A \implies B, \quad A \iff B, \quad A = \{z \in \mathbb Z \mid z = \bar z \};
\end{equation}
\begin{equation}
    f: X \to Y, \quad f: x \overset{F}{\mapsto} 5x \cos(\tfrac{\pi x}{2})
\end{equation}
\begin{equation}
    \frac{\cancel{\sqrt 2} \sin(x + 2)}{\cancel{2 \cdot \cos(\sfrac{\pi}{4})} \sin(x)} = \frac{\sin(x + 2)}{\sin(x)}, \quad \cancelto{0}{\sin(0)} \equiv 0.
\end{equation}

\section{Определения и теоремы}

В ГОСТе не представлены инструкции по оформлению определений и теорем. Однако в мировой практике принято следующее стилевое решение, которое можно использовать и в дипломной работе:

\begin{Df}[Метрическое пространство]\label{df:example}
    Метрическим пространством называют пару $(S; \rho)$, где для функции-\textit{метрики} $\rho: S \to \mathbb R^+$, если верно следующие три условия:
    \begin{enumerate}
        \item $a \in S : \rho(a, a) = 0$,
        \item $a, b \in S : \rho(a, b) = \rho(b, a)$,
        \item $a, b, c \in S : \rho(a, b) + \rho(b, c) \ge \rho(a, c)$.
    \end{enumerate}
\end{Df}

\begin{Th}[Хаусдорф]\label{th:example}
    В полном метрическом пространстве множество является компактом тогда и только тогда, когда оно замкнуто и вполне ограничено.
\end{Th}
\begin{proof}
    Текст доказательства.
\end{proof}

\begin{Ex}
    Множество вещественных чисел $\mathbb R$ с заданной на нём метрикой $\rho(a, b) = |a - b|$ формируют метрическое пространство.
\end{Ex}
\begin{proof}
    Следует из определения \ref{df:example}.
\end{proof}

Аналогично картинкам и таблицам, на определения, теоремы и другие блоки можно ссылаться при помощи команды \verb|\ref|: например, на теорему \ref{th:example}.
